%%%%%%%%%%%%%%%%%%%%%%%%%%%%%%%%%%%%%%%%%%%%%%%%%%%%%%%%%%%%%%%%%%%%%%%%%%%%%%

\documentclass{l3deliverable}

%%%%%%%%%%%%%%%%%%%%%%%%%%%%%%%%%%%%%%%%%%%%%%%%%%%%%%%%%%%%%%%%%%%%%%%%%%%%%%

\usepackage{graphicx}%
\usepackage{url}%

%%%%%%%%%%%%%%%%%%%%%%%%%%%%%%%%%%%%%%%%%%%%%%%%%%%%%%%%%%%%%%%%%%%%%%%%%%%%%%
%% See D1 for an example of how to integrate sub version revision
%% numbers into a LaTeX document.
%

%%%%%%%%%%%%%%%%%%%%%%%%%%%%%%%%%%%%%%%%%%%%%%%%%%%%%%%%%%%%%%%%%%%%%%%%%%%%%%
%% Check these macro values for appropriateness for your own document.

\title{Change Report}

\author{Ross Adam \\
        Andrew Gardner \\
        Nicole Kearns \\
        Mamas Nicolaou\\
	Asset Sarsengaliyev\\}

\date{28th February 2013}

\deliverableID{D8.4}
\project{PSD3 Group Exercise 8}
\team{V}
\version{0.1}


%%%%%%%%%%%%%%%%%%%%%%%%%%%%%%%%%%%%%%%%%%%%%%%%%%%%%%%%%%%%%%%%%%%%%%%%%%%%%%

\begin{document}

%%%%%%%%%%%%%%%%%%%%%%%%%%%%%%%%%%%%%%%%%%%%%%%%%%%%%%%%%%%%%%%%%%%%%%%%%%%%%%

\maketitle

\tableofcontents

\newpage

%%%%%%%%%%%%%%%%%%%%%%%%%%%%%%%%%%%%%%%%%%%%%%%%%%%%%%%%%%%%%%%%%%%%%%%%%%%%%%
%% Standard section for all documents

\section{Introduction}

Team V's a change report describing our modifications to our internship management system.

\section{Report}

In order to include the new requirements in our internship management system, we had to make several changes to our implementation. The changes applied to our internship management system are:\\

\begin{tabular}{|c|c|c|}
\hline \textbf{ID} &\textbf{Package/Class} & \textbf{Changes Applied}\\
\hline 1 & uk.ac.glasgow.internman.users.StudentUser & Return type for getInternship() \\
\hline 2 & uk.ac.glasgow.internman.impl.InternManStub & approveAcceptedOffer() parameters \\
\hline 3 & uk.ac.glasgow.internman.ui.ApproveOfferCommand & Changed parameters to reflect Change ID 2\\
\hline 4 & uk.ac.glasgow.internman.ui.ViewStudentDetailCommand & Changed to iterate over internship details \\
\hline 5 & uk.ac.glasgow.internman.ui.InternManStub & selectStudent()\\
\hline
\end{tabular}

\subsection{Change ID: 1}

In our previous implementation, a student could only accept one internship. However, according to the new requirements, a student must be able to accept more than one internship. In order to meet these requirements, we had to change the getInternship() method from an Object Internship to a List<Internship>. This allows the system to store a list of all internships that a student has accepted rather than a single internship per student. As we had changed this method in the java class, we then had to change the interface to reflect this change.\\

\subsection{Change ID: 2}

In our previous implementation, when the course coordinator was approving an accepted internship, he simply had to pass in the students matriculation number. However, as the student can now have multiple accepted internships, we had to change the parameters for the ApproveAcceptedOffer() method of the facade to pass in an Internship index, which specifies what internship the course coordinator was approving along with the student's matriculation number.

\subsection{Change ID: 3}

Previously, within the ApprovedOfferCommand, when calling the ApproveAcceptedOffer() method, it was simply passing in the students matriculation number. We modified this to pass in the students Internship ID, as well as the students matriculation number. \\
Due to the changes in Change ID 2, a number of commands in uk.ac.glasgow.internman.ui/.ApproveOfferCommand were malfunctioning. To solve this we had to apply some modifications to some methods of the class in order to get them to work with the changes that were made to the facade. Namely, we created an integer internshipID which is reading in args[1], changed the approveAcceptedOffer() parameters to include the internship ID and modifiied the way in which the system retrieves an internship.\\


\subsection{Change ID: 4}

For our previous implementation, a student was only able to accept one internship. Therefore in the ViewStudentDetailCommand, the system was only storing the details of an internship once. If a student accepted a different internship, the previous details were simply overwritten. However, as the new requirements state that a student should be able to accept more than one internship, the system must be able to store these details for each internship accepted. 
To deal with this requirement, we had to loop over all internships the student had applied for and store these details for each of them.\\

\subsection{Change ID: 5}

In our previous implementation, only the course-coordinator was allowed to view a students details. However, the new requirements stated that a student should be able to view their own details as these are stored in the system. To change the implementation to suit this requirement, we modified the selectStudent() method within the facade. That means that a student can now request their details by calling the method selectStudent() and providing their matriculation number. The method now simply checks to see if the GUID of the user currently logged in to the system matches the matriculation number provided to the method. This means that a student can view only their own details.\\

\end{document}
