%%%%%%%%%%%%%%%%%%%%%%%%%%%%%%%%%%%%%%%%%%%%%%%%%%%%%%%%%%%%%%%%%%%%%%%%%%%%%%

\documentclass{l3deliverable}

%%%%%%%%%%%%%%%%%%%%%%%%%%%%%%%%%%%%%%%%%%%%%%%%%%%%%%%%%%%%%%%%%%%%%%%%%%%%%%

\usepackage{graphicx}%
\usepackage{url}%
\usepackage{usecasedescription}%

%%%%%%%%%%%%%%%%%%%%%%%%%%%%%%%%%%%%%%%%%%%%%%%%%%%%%%%%%%%%%%%%%%%%%%%%%%%%%%
%% See D1 for an example of how to integrate sub version revision
%% numbers into a LaTeX document.
%

%%%%%%%%%%%%%%%%%%%%%%%%%%%%%%%%%%%%%%%%%%%%%%%%%%%%%%%%%%%%%%%%%%%%%%%%%%%%%%
%% Check these macro values for appropriateness for your own document.

\title{Draft Submission}

\author{Ross Adam \\
        Andrew Gardner \\
        Nicole Kearns \\
        Mamas Nicolaou\\
	Asset Sarsengaliyev\\}

\date{30th January 2013}

\deliverableID{D6}
\project{PSD3 Group Exercise 2}
\team{V}
\version{0.1}

%%%%%%%%%%%%%%%%%%%%%%%%%%%%%%%%%%%%%%%%%%%%%%%%%%%%%%%%%%%%%%%%%%%%%%%%%%%%%%

\begin{document}

%%%%%%%%%%%%%%%%%%%%%%%%%%%%%%%%%%%%%%%%%%%%%%%%%%%%%%%%%%%%%%%%%%%%%%%%%%%%%%

\maketitle

\tableofcontents

\newpage

%%%%%%%%%%%%%%%%%%%%%%%%%%%%%%%%%%%%%%%%%%%%%%%%%%%%%%%%%%%%%%%%%%%%%%%%%%%%%%
%% Standard section for all documents

\section{Introduction}

\subsection{Identification}
Final report for the internship management project, containing the project plan, requirements specification, a report and demonstration of the prototype.

\subsection{Related Documentation}

PSD3 Group Exercise Description \url{http://fims.moodle.gla.ac.uk/file.php/128/coursework/psd3-ge-1-rev3278.pdf}\\

Deliverables Template \url{http://fims.moodle.gla.ac.uk/file.php/128/coursework/templates.zip}\\

PSD3 Course Notes \url{http://fims.moodle.gla.ac.uk/file.php/128/lecture-notes/notes-r3275.pdf}\\

\subsection{Purpose and Description of Document}
The purpose of this document is to give a detailed description of our
system prototype. This report includes a description of the scope and
design of the system and an evaluation of our prototype.
\subsection{Document Status and Schedule}

\begin{center}{
\begin{tabular}{|c|c|c|c|}
\hline \textbf{Date} &\textbf{ Change} & \textbf{Version} &\textbf{Author}\\ 
\end{tabular} }
\end{center}

%%%%%%%%%%%%%%%%%%%%%%%%%%%%%%%%%%%%%%%%%%%%%%%%%%%%%%%%%%%%%%%%%%%%%%%%%%%%%%

\section{Component Diagram}

%%%%%%%%%%%%%%%%%%%%%%%%%%%%%%%%%%%%%%%%%%%%%%%%%%%%%%%%%%%%%%%%%%%%%%%%%%%%%%

\section{State Diagrams}

\subsection{Rationale} \\

The following three diagrams were created to illustrate how entities within the application change over time. The three entities are Student, Internship and Advert. These entities were identified by the ``status'' variable in their specification. \\

\subsection{Student}\\
\textbf{Description} \\

Student was choosen because it displayed 4 distinct states: Approved, Accepted, Withdrawn and Pending. These were identified from the ``View Student Details'' use case. However the finding other use cases to justify this proved difficult as none could be found that implied a change in the Student entity state. Therefore the following diagram was implemented in a manner deemed logical by the team.

\textbf{Diagram}







%%%%%%%%%%%%%%%%%%%%%%%%%%%%%%%%%%%%%%%%%%%%%%%%%%%%%%%%%%%%%%%%%%%%%%%%%%%%%%

\section{Class Diagrams}

%%%%%%%%%%%%%%%%%%%%%%%%%%%%%%%%%%%%%%%%%%%%%%%%%%%%%%%%%%%%%%%%%%%%%%%%%%%%%%

\section{API Specification}

%%%%%%%%%%%%%%%%%%%%%%%%%%%%%%%%%%%%%%%%%%%%%%%%%%%%%%%%%%%%%%%%%%%%%%%%%%%%%%

\section{Acceptance Tests}


