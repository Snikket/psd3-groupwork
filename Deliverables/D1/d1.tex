
%%%%%%%%%%%%%%%%%%%%%%%%%%%%%%%%%%%%%%%%%%%%%%%%%%%%%%%%%%%%%%%%%%%%%%%%%%%%%%
% This is a template for constructing your project plan document, but
% also to show the use of the l3deliverable class. Use pdflatex and
% bibtex to process the file, creating a PDF file as output (there is
% no need to use dvips when using pdflatex).
%
% Several meta data commands have been implemented to collect
% information such as deliverable identifier, project name etc (see
% below the \date command.

\documentclass{l3deliverable}

%%%%%%%%%%%%%%%%%%%%%%%%%%%%%%%%%%%%%%%%%%%%%%%%%%%%%%%%%%%%%%%%%%%%%%%%%%%%%%
% You can use the svn-multi package to automatically insert version
% control information into your document (an example of how to do this
% is shown below).  Make sure to set the 'svn:keywords' subversion
% property to 'Id' for the source file, for example, type:
%
% svn propset svn:keywords 'Id' d1.tex
%
% in the same directory as your 'd2.tex' file. 
%
% The information between the two $$ will now be updated when you next
% commit the file to your SVN repository.
%
% You can of course, just use this field to insert manual version
% information, e.g. 1.2, 1.2.1 ... instead.

%%%%%%%%%%%%%%%%%%%%%%%%%%%%%%%%%%%%%%%%%%%%%%%%%%%%%%%%%%%%%%%%%%%%%%%%%%%%%%

\usepackage{url}

%%%%%%%%%%%%%%%%%%%%%%%%%%%%%%%%%%%%%%%%%%%%%%%%%%%%%%%%%%%%%%%%%%%%%%%%%%%%%%
%% Check these macro values for appropriateness for your own document.

\title{Team Organisation}

%%authors
\author{Ross Adam\\
     	Andrew Gardner\\
     	Nicole Kearns\\
 	Mamas Nicolaou\\
    	Asset Sarsengaliyev\\
}

%%release date 
\date{10 January 2009}

\deliverableID{D1}
\project{PSD3 Group Exercise 1}
\team{V}

%%%%%%%%%%%%%%%%%%%%%%%%%%%%%%%%%%%%%%%%%%%%%%%%%%%%%%%%%%%%%%%%%%%%%%%%%%%%%%

\begin{document}

%%%%%%%%%%%%%%%%%%%%%%%%%%%%%%%%%%%%%%%%%%%%%%%%%%%%%%%%%%%%%%%%%%%%%%%%%%%%%%

\maketitle

%%%%%%%%%%%%%%%%%%%%%%%%%%%%%%%%%%%%%%%%%%%%%%%%%%%%%%%%%%%%%%%%%%%%%%%%%%%%%%
%% Standard section for all documents

\section{Introduction}

\subsection{Identification}

Organisation Plan document for Team V’s PSD3 Project.

\subsection{Related Documentation}

PSD3 Group Exercise Description \url{http://fims.moodle.gla.ac.uk/file.php/128/coursework/
psd3-ge-1-rev3278.pdf}\\
Deliverables Template \url{http://fims.moodle.gla.ac.uk/file.php/128/coursework/templates.zip}\\
PSD3 Course Notes \url{http://fims.moodle.gla.ac.uk/file.php/128/lecture-notes/notes-r3275.pdf}\\
 
\subsection{Purpose and Description of Document}

The purpose of this document is to detail and explain the organisational structure of Team V’s PSD3
Group Project. This includes role assignment and the risks associated with the organisational structure
chosen.

\subsection{Document Status and Schedule}

\begin{center}{
\begin{tabular}{|c|c|c|c|}
\hline \textbf{Date} &\textbf{Change} & \textbf{Version} & \textbf{Author}\\ 
\hline 24/09/2012 & Began Draft & 0.1 & All \\ 
\hline 25/09/2012 & Initial Draft Completed & 0.2 & All \\ 
\hline 26/09/2012 & Finalised for Submission & 0.3 & All\\ 
\hline 27/09/2012 & \textbf{Draft Submission Deadline} & 1.0 & All\\ 
\hline ... & Revision & & All\\ 
\hline 29/11/2012 & \textbf{Final Submission Deadline} & & All\\ 
\hline 
\end{tabular} }
\end{center}

%%%%%%%%%%%%%%%%%%%%%%%%%%%%%%%%%%%%%%%%%%%%%%%%%%%%%%%%%%%%%%%%%%%%%%%%%%%%%%

\section{Roles}

\begin{center}{
\begin{tabular}{|c|c|}
\hline \textbf{Name} &\textbf{Roles}\\ 
\hline Ross Adam & Project Manage, Test Manager, Quality Assuror\\ 
\hline Andrew Gardner & Chief Architect, Customer Liaison, Quality Assuror  \\ 
\hline Mamas Nicolaou & Toolsmith, Librarian, Quality Assuror\\ 
\hline Nicole Kearns & Secretary, Test Manager, Quality Assuror\\ 
\hline Asset Sarsengaliyev & \textbf{ENTER ROLES!}\\
\hline 
\end{tabular} }
\end{center}
%%%%%%%%%%%%%%%%%%%%%%%%%%%%%%%%%%%%%%%%%%%%%%%%%%%%%%%%%%%%%%%%%%%%%%%%%%%%%%

\section{Authority}

The Project Manager role was assigned to Ross Adam. Important group decisions will be democratically
discussed within the group with the Project Manager enforcing the consensus and ensuring
deadlines are met.

%%%%%%%%%%%%%%%%%%%%%%%%%%%%%%%%%%%%%%%%%%%%%%%%%%%%%%%%%%%%%%%%%%%%%%%%%%%%%%

\section{Communication}

Meetings will be held at least once a week, likely within the University’s library or the Level 3 Computing
Lab. Subsequent meeting arrangements will be discussed and finalised at the end of each
meeting.

%%%%%%%%%%%%%%%%%%%%%%%%%%%%%%%%%%%%%%%%%%%%%%%%%%%%%%%%%%%%%%%%%%%%%%%%%%%%%%

\section{Information Management}

The librarian is repsonsible for maintaining the documents and information that pertain to the project.
We intend to use..\\

Mahara: \textit{Utilising a private forum as a central document and information index.}\\
GitHub: \textit{Online SCM hosting for our codebase and Latex documents.}\\
Facebook: \textit{Informal progress updates, idea sketching and general conversation.}\\


%%%%%%%%%%%%%%%%%%%%%%%%%%%%%%%%%%%%%%%%%%%%%%%%%%%%%%%%%%%%%%%%%%%%%%%%%%%%%%

\section{Organisational Risks}

The role based model suggests less risks than with alternatives but if someone was to fall behind or
become absent then their role within the team would be compromised. If not immediately addressed
this could threaten the project’s integrity.

%%%%%%%%%%%%%%%%%%%%%%%%%%%%%%%%%%%%%%%%%%%%%%%%%%%%%%%%%%%%%%%%%%%%%%%%%%%%%%

\appendix

\section{Glossary}

\textbf{PSD3} Professional Software Development 3\\
\textbf{SCM} Source Control Management\\

\section{Another appendix}

\begin{center}{
\begin{tabular}{|c|c|}
\hline \textbf{Meeting Plan}\\ 
\hline
\hline Day & Monday \\
\hline Time & 12:00\\
\hline Location & Boyd Orr Lab 720\\
\hline Attendees & All\\
\hline Previous Minutes\\
\hline Current Agenda & TBC\\
\hline Goals & TBC\\
\hline Arrange next meeting & &TBC \\
\hline
\end{tabular} }
\end{center}

\hline 24/09/2012 & Began Draft & 0.1 & All \\ 

\end{document}

%%%%%%%%%%%%%%%%%%%%%%%%%%%%%%%%%%%%%%%%%%%%%%%%%%%%%%%%%%%%%%%%%%%%%%%%%%%%%%
